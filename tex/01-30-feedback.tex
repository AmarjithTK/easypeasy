\documentclass[easypeasy.tex]{subfiles}
\chapter{Feedback}
\begin{document}

The war isn't against PMOers, but the porn industry and trap and I wage it for the simple reason that I enjoy it. Every time I hear about a PMOer escaping from the prison I get a feeling of immense pleasure. But this pleasure hasn't been without considerable frustration, which is mainly caused by two catagories of PMOer. In spire of the warning in the previous chapter, I am continually suprised by the number of PMOers who find it easy to stop, yet later get hooked and find they can't succeed the next time.

It's like finding someone up to their neck in a swamp and about to go under. You help pull them out and they're grateful, then, six months later dives stright back into the swamp. PMOers who find it easy to stop and start again pose a special problem, however when you get free \textit{PLEASE, PLEASE, DON'T MAKE THE SAME MISTAKE.} They believe such people start again because they are still hooked and are missing the dopemine. In fact, they find stopping so easy that they lose their fear of PMOing. They think, "I can have an odd PMO session, even if I do get hooked again, I'll find it easy to stop again."

I'm afraid it just doesn't work that way, it's easy to stop PMOing but impossible to control the addiction. The one thing essential to stopping PMO is not to PMO. 

The other catagory of PMOers that cause me frustration are those who are just too frightened to make the attempt to stop, or when they do, find it a great struggle. THe main difficulties appear to be the following.
\begin{description}
  \item [Fear of failure.] There is no disgrace in failure, but not to try is plain stupidity. Look at it this way, you're hiding from nothing. The worst thing that can happen is that you fail, in which case you are no worse off than you are now. Just think how wonderful it would be to succeed. If you don't make the attempt, you have already guarenteed failure.

  \item [Fear of pain and being miserable.] Don't worry about it, just think: what awful thing could happen to you if you never PMOed again? Absolutely nothing. Terrible things will happen if you do. Re-read the notes on Pascal's Wager. In any case, the panic is caused by dopamine and will soon be gone. The greatest gain is to be rid of that fear. Do you really believe that PMOers are prepared to have fading penetrations, unreliable sexual performance or the pleasure of orgasm they get from porn? If you find yourself getting panicky, deep breathing will help. If you are with other people and they are getting you down, escape from them and go to the garage, an empty office or somewhere.

  If you feel like crying, don't be ashamed. Crying is nature's way of relieving tension. No-one has ever had a good cry without feeling better afterwards. One of the awful things we do to young boys is teaching them not to cry. You can see them trying to fight back the tears, but watch the jaw grinding away. We teach ourselves not to show any emotions, but we're meant to, not bottling them up inside. Scream, shout or have a tantrum. Kick something. Regard your struggle as a boxing match that you cannot lose. Nobody can stop time, every moment that passes that little monster inside you is dying. Enjoy your inevitable victory. 

  \item [Not following the instructions.] Incredibly, some PMOers say that the method didn't work for them. They then describe how they ignored not only one instruction but practically all of them. For clarity, these are summarised as a checklist at the end of this chapter.

  \item [Misunderstanding instructions.] The chief problems appearing to be these:

    \begin{description} 
    \item["I can't stop thinking about porn."]
      Of course you can't and if you try, you will create a phobia and become miserable. It's like trying to get to sleep at night; the more you try, the harder it becomes. It doesn't matter if you think about porn and PMO for ninety percent of your life, it's what you're thinking that's important. If you're thinking "Oh, I love to PMO" or "When will I be free?" you'll be miserable. If you're instead thinkng "YIPPEE! I am free" you'll be happy.

      \item["When will the little porn monster die?"]
        The dopamine flush leaves your body very rapidly. But it's impossible to tell when your body will cease suffering from the slight physical sensation of dopamine withdrawal. That empty, insecure feeling is identical to normal hunger, depression or stress. All PMO does is to increase the level of it. This is why PMOers who stop using the willpower method are never quite sure if they've kicked it, even after the body has ceased to suffer from the dopamine surge withdrawal. If they suffer normal hunger or stress, their brain tells them this is a valid reason to claim their entitled PMO. The point being that you don't have to wait for the craving to go, it's so slight that we don't even know it's there, only knowing it as a feeling of wanting. When you leave the dentist after the final session, do you wait for your jaw to stop aching? Of course you don't, you get on with life. Even though your jaw's still aching, you are elated.

      \item [Waiting for the 'moment of revelation'.] If you wait for it, you are just causing another phobia, I once stopped for three weeks on the willpower methid.

      Chatting with an old friend, he asked me how I was getting on.
      I said "I've survived three weeks."
      He said "What do you mean, you've survived three weeks?"
      I said "I've gone three weeks without a PMO."
      He said, "What are you going to do? Survive the rest of your life? What are you waiting for, you've done it. You're a non-PMOer."

      I thought, "He's absolutely right, what am I waiting for?" Unfortunately, due to a lack of understanding of the trap, I was soon back in but the point was noted. You become a non-PMOer when you close your browser. The important thing is to be a happy non-PMOer from the start.

      \item ["I am still craving porn."] Then you're being very stupid. How can you claim that you want to be a non-PMOer and then say that you want porn? That's a contradiction. If you say that you want to PMO, you're saying that you want to be a PMOer. Non-PMOers don't want to visit the disgusting tube sites. You already know what you want to be, so stop punishing yourself.

      \item ["I've opted out of life."] Why? All you have to do is stop killing yourself and start energising yourself. You don't have to stop living in the slightest. It's as simple as this, for the next couple of days you will have a slight trauma in your life. Your body will suffer the almost imperceptible aggravation of withdrawal from your demands and claims for a dopamine surge. Now, bear this in mind: you are no worse off than you were. This is what you have been suffering or the whole of your life, every time you've been asleep, in church, the supermarket or library. It didn't seem to bother you when you were a PMOer and if you don't stop, you'll go on suffering this distress for the rest of you life.

      PMO and orgasms don't make meals, drinks or social occasins, they deprive you of them. Even while your body is still craving dopamine surges, meals and social occasions are marvelous. Life is marvellous, go to social functions, even if there are naked dancers there. Remember you're not being deprived, they are. Every one of them would love to be in your position, if only they knew. Enjoy being the prima donna and centre of attention. Stopping PMO is a wonderful conversation point, taking a secret pleasure they cannot. Your friends and peers will be surprised to see that you, a shying and tired looking fellow is now looking happy and cheerful. You'll be enjoying life right from the start, there's no need to envy pick up artists at parties, they'll be envying you - if only they knew.

     \item ["I am miserable and irritable."] This is failure to follow instructions. Find out which one it is. Some people understand and believe everything written but still start off with a feeling of doom and gloom, as if something terrible was happening. You are not only doing what you'd like to do, but what every PMOer on the planet would like to do. With any method of stopping the ex-PMOer is trying to achieve a certain frame of mind, so whenever they think about PMO they say to themselves "YIPPEE! I'M FREE!" If that's your objective, why wait? Start off in that frame of mind and never lose it. The rest of the book is designed to make you understand why there is no alternative.

\end{description}
\end{description}

\section{The Checklist}

  If you follow these instructions, you cannot fail:
\begin{enumerate}
    \item Make a solemn vow that you will never, ever, go online to visit your harem OR settle for static pictures OR make peace with erotic graphics OR anything that contains supernormal stimulai and stick to your vow.

    \item Get this clear in your mind: There is absolutely nothing to give up. By that, it isn't meant that you will be better off as a non-PMOer (you've known this all along); nor that although there is no rational reason why you PMO, getting some pleasure or crutch from it otherwise you wouldn't do it. What is meant is there is \textit{no genuine pleasure or crutch in PMOing}. It's just an illusion, like banging your head against a wall to get pleasure when you stop.

    \item There is no such thing as a confirmed PMOer. You're just one of the millions who have fallen for the subtle trap. Like millions of other ex-PMOers who once thought they couldn't escape, you have escaped.

    \item If at any time in your life you were to weight up the pros and cons of PMOing, the overwhelming conclusion would always be "Stop doing it. You are a fool" Nothing will ever change that. It has always been that way and always will be. Having made what you know to be the correct decision, don't ever torture yourself by doubting. Going through Pascal's Wager with no chance of loss, high chances of gains and high chances of avoiding losses perfect applies to PMO.

    \item Don't try not to think of porn, or worry that you are thinking about it constantly. Whenever you do think about it, whether it be today, tomorrow or the rest of your life, think "YIPPEE! I'M A NON-PMOer!"

    \item \textbf{Do not} use any form of substitute. \textbf{Do not} keep your laptop next to you while you sleep. \textbf{Do not} avoid plays, movies or magazines. \textbf{Do not} change your lifestyle in any way purely because you've stopped. If you follow the above instructions, you will soon experience the 'moment of revelation', but:

    \item Don't wait for the 'moment of revelation' to come. Just get on with your life. Enjoy the high cope with the lows. You will find in no time at all the moment will arrive.
\end{enumerate}
\end{document}
