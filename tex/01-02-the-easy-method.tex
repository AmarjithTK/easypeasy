\documentclass[easypeasy.tex]{subfiles}
\chapter{The Easy Method}

\begin{document}
The object of this book is to get you into the frame of mind in which, instead of the usual method of stopping whereby you start off the feeling that you are climbing Mount Everest and spend the next few weeks craving and feeling deprived, you start right away with a feeling of elation, as if you had been cured of a terrible disease. From then on, the further you go through life, the more you will look at this period of time and wonder how you ever used any porn in the first place. You will look at other porn users with pity, as opposed to envy.

Provided that you are someone who had never become addicted (you are reading for your significant other) or had quit (or is in the fasting days of a "porn diet"), it is essential to keep using until you have finished the book completely. This may appear to be a contradiction, and this instruction to continue masturbating to porn causes more objection than any other, but as you read further your desire to use porn will gradually be reduced. \textbf{Take this instruction seriously: Attempting to quit early will not benefit you.}

Many don't finish the book because they feel they have to give something up, some even deliberately only reading one line per day in order to postpone the evil event. Look at it this way, what have you got to lose? If you don't stop at the end of the book, you're no worse off than you are now. It's by definition a Pascal's Wager, a bet taken where you have nothing to lose and high chances of large gains.

Incidentally, if you haven't watched porn for a few days or weeks, but aren't sure whether you're a porn user, ex-user, or a non-user, then don't use porn to masturbate while you read. In fact, you're already a non-user, but we have to let your brain catch up with your body. By the end of the book, you'll be a happy non-user. EasyPeasy is the complete opposite of the normal method, where one lists the considerable disadvantages of porn and says: \\
\textit{"If only I can go long enough without porn, eventually the desire will go and I can enjoy life again, free of slavery."} \\
This is the logical way to go about it, with thousands stopping every day using this method. However, it's very difficult to succeed for the following reasons:

\begin{description}
  \item [Stopping PMO isn't the real problem.] Every time you finish your session, you've stopped using it. You may have powerful reasons on the first day of your once-in-four porn diet to say \textit{``I don't want to use porn, or even masturbate any more.''} All users do, and their reasons are more powerful than you can possibly imagine. The real problem is day two, ten, or ten-thousand where in a weak moment you'll have `just one peek', want another, and suddenly you're an addict again.

  \item [Awareness of the health risks generates more fear, making it more difficult to stop.] Tell a user it's destroying their virility and the first thing they'll do is reach for something to surge their dopamine: a cigarette, alcohol, or even firing up the browser to search for porn.

  \item [All reasons for stopping actually make it harder.] This is due to two reasons. First, we're continually being forced to give up our 'little friend' or some prop, vice, or pleasure (whichever way the user perceives it). Second, they create a "blind". We do not masturbate for the reasons we should stop. The real question is, why do we want or need to do it?
\end{description}

With EasyPeasy, we (initially) forget the reasons we'd like to stop, face the porn problem and ask ourselves the following questions:

\begin{enumerate}
\item What is porn doing for me?
\item Am I actually enjoying it?
\item Do I really need to go through life sabotaging my mind and body?
\end{enumerate}

The beautiful truth is that \textit{all porn} does absolutely nothing for you whatsoever. Let me make it quite clear, it's not that the disadvantages of being a user outweigh the advantages, it's that there are \textbf{zero} advantages to looking at internet porn.

Most users find it necessary to rationalise why they use porn, but the reasons they come up with are all fallacies and illusions.

The first thing we are going to do is to remove these fallacies and illusions. In fact, you will realize that there is nothing to give up. Not only is there nothing to give up but there are marvelous, positive gains from being a non-PMOer, and well-being and happiness are only two of these gains. Once the illusion that life will never be quite as enjoyable without porn is removed, once you realize that not only is life just as enjoyable without it but infinitely more so, once the feeling of being deprived or of missing out is eradicated, then we can go back to reconsider the well-being and happiness—and the dozens of other reasons for quitting porn. These realizations will become positive additional aids to help you achieve what you really desire to enjoy the whole of your life free from the slavery of porn addiction.
\end{document}
