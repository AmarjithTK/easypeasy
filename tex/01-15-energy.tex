Energy

Most PMOers are aware of the effects that the progressive process of PMO, leading to novelty and escalation seeking, has on their brain's reward and sexual system. However, they are not aware of the effect it has on their energy level.

One of the subtleties of the PMO trap is the effects that it has on us, both physical and mental, happen so gradually and imperceptibly that we are not aware of them and regard them as normal. The effect is similar to that of bad eating habits, we look at people who are grossly overweight and wonder how they could have possibly allowed themselves to reach that state. But suppose that it happened overnight, you went to bed trim, rippling with muscles and not an ounce of fat on your body, awaking to find yourself fat, bloated and pot-bellied. Instead of waking up feeling fully rested and full of energy, you wake up feeling miserable, lethargic and barely able to open your eyes.

You would be panic stricken, wondering what awful disease you had contracted overnight, yet the disease is exactly the same. The fact that it took you twenty years to get there is irrelevant. PMOing is the same, if it was possible to immediately transfer your mind and body to give you a direct comparison on how you would feel having stopped PMO for just three weeks, it would be all that would be need to be done to convince you. You would ask if you would really feel this good, or what it really amounts to, "had I really sunk that low?". You wouldn't just feel healthier with more energy, but far more confidence and a heightened ability to concentrate.

The lack of energy, tiredness and everything related to it is nicely swept under the rug of 'getting older'. Friends and colleagues, also living sedentary lifestyles further compound the normalisation of this behaviour. The belief that energy is the exclusive prerogative of children and teenagers and that old age begins in your twenties is another symptom of the brainwashing. Being unaware of eating habits and not paying attention to your eating habits is as a result of the compounding effects of dopamine desensitisation.

Shortly after stopping PMO, the foggy and muggy feeling will leave you. The point is, with PMO you are always debiting your energy and in that process, tampering with the neurological chemistry of your reward circuit. Unlike quitting smoking, where the return of your physcal and mental health is only gradual, quitting PMO gives you excellent results from day one. Killing the 'little monster' and closing the water slides takes a little bit of time, but recovering your reward centre is nothing like as slow as the slide into the pit. If you are going through the trauma of the willpower method, any health or energy gains will be obliterated by the depression you will be going through. Unfortunately, it's not possible for EASYPEASY to immediately transfer you into your mind in three weeks time, but you can! You know instinctively that what you are being told is correct, all you need to do is \textbf{use your imagination!}
