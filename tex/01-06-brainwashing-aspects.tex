\documentclass[easypeasy.tex]{subfiles}
\chapter{Brainwashing Aspects}
\begin{document}

The big monster of the porn trap is bred through culmination of many aspects including societal forces, media, peers and the users own internal narrative. Failures to deconstruct the various fallacies when using the willpower method eventually leads the user back into the trap through feelings of deprivation. Deconstruction of the imagined value of porn is crucial for success, allowing you to see where you're being robbed!

Of importance to note is the link between brainwashing and fear. It's fear of feeling \textit{\textbf{future withdrawal pangs}} that create the pangs. Fear is the pang itself. Think when you've had withdrawal symptoms such as sweaty palms, shortness of breath, sleeping problems and inability to think straight. Now think of similar situations when you've had those feelings, job interviews, nerves around an attractive person, public speaking, ect. These are the same anxious feelings the fear causes. Simply put, how can a physical drug still hook people months after stopping? It must be mentally, correct?

\section{Stress}

Not only great tragedies in life but also minor stresses drive the user one step further into the forbidden area previously excluded. Stresses includes socialising, phone calls, anxieties of the housewife with young children and many others. Taking phone calls as an example, particularly for a businessperson. Most calls aren't from satisfied customers or your boss congratulating you, there's some sort of aggravation. Coming home to mundane family life of kids screaming and their partners emotional demands causes the user, if they aren't already doing so, to fantasise the relief of porn promised that night. Unconsciously suffering withdrawal pangs, weakened destressors unprepared for additional aggravation. Partially relieving the pangs at the same time as normal stress, the total is reduced and the user gets a temporary boost. The boost isn't an illusion, the user does genuinely feel better than before, but they're more tense than they would be as a non-user.

The following example isn't designed to shock you, EasyPeasy promises no such treatment, but is to emphasise that porn destroys your nerves rather than relaxing them.

Try to imagine getting to the stage where you're unable to be aroused, even with a very sexy and attractive partner. For a moment, pause and try to visualise life where a very lovely and charming person has to compete and fail with the virtual porn stars who occupy your 'harem' to get your attention. Imagine the frame of mind of a person, who issued with that warning, continues using and dies without ever having real sex with this charming and willing partner. It's easy to dismiss these people as weirdos but stories like these aren't fakes, this is what the awful novelty of the porn drug does to your brain. The more you go through life, the more courage is sapped and the more you're deluded into the believing porn is doing the opposite.

Have you ever been overtaken by panic when out of the blue the WiFi stops working or is too slow? Non-users don't suffer from it with the internet porn drug causing that feeling. As you go through life, it systematically destroys your nerve and courage, leaving DeltaFosB to form a powerful neural water slide in it's wake, progressively destroying your ability to say no. By the stage where virility has been killed, the user believes porn is their new partner and is unable to face life without it.

Internet porn isn't relieving your nerves, it's slowly destroying them. One of the great gains of breaking the addiction is the return of your natural confidence and self assurance.

There's no need to self-rate based on your ability to satisfy a partner, this isn't freedom. But this freedom cannot be obtained by continuing to grease the dopamine water slide in ways that undercut your happiness and libido by repeating the same destructive behaviour.

\section{Boredom}

If you're like many people as soon as you get into bed you're already on your favorite porn site, probably already forgetting until you were reminded. It's become second nature. Porn relieving boredom is another fallacy as boredom is a frame of mind. The only time this happens is when you've been deprived for a long time or are trying to cut down.

The actual situation is this, when you're addicted to the supernormal pull of internet porn and then try to abstain it feels like there's something missing. If you have something to occupy your mind that isn't stressful, you can go for long periods of time without being bothered by the absence of the drug. However, when you're bored there's nothing to take your mind off it, so you feed the monster. When you're indulging yourself and not trying to stop or cut down, even firing up private browsing becomes subconscious. The user performs this ritual automatically, if trying to remember sessions during the last week, only able to remember a small proportion of them, like the very last one or the session after a long abstinence.

The truth is that porn increases boredom indirectly because orgasms make you feel lethargic and instead of undertaking an energetic activity, users tend to prefer lounging around, bored and relieving their withdrawal pangs. Countering the brainwashing is important because users tend to view porn when they're bored, our brains wired to interpret porn as interesting. Similarly, we've also been brainwashed into believing sex, even bad sex, aids relaxation. It's a fact that when sad or under stress, couples want to have sex. In the absence of discrimination between amative and propagative sex, watch how quickly you want to get away from each other after the mandatory orgasm is achieved. If the couple had just decided to hug, speak or cuddle and go to sleep, they'd have felt relieved.

\section{Concentration}

Masturbation and sex don't help concentration, when you're trying to concentrate you automatically try and avoid distractions. Therefore, when a user wants to concentrate, they don't even think, automatically opening the browser, feeding the little monster and partially ending the craving. Getting on with the matter at hand, they've already forgotten they've viewed porn. After years of dopamine flooding the brain changes affect abilities such as accessing information, planning and impulse control.

You're also driven to provide novelty for the next session as the same stuff won't generate enough dopamine and opiates so you'll have to roam the internet streets for novelty, fighting the pull to cross the line towards shocking material, this in turn generating more stress and leaving you unfulfilled after finishing.

Concentration is also adversely affected due to dopamine receptors being culled due to natural tolerance to the large surges, reducing the benefit of smaller dopamine boosts from natural de-stressors. Your concentration and inspiration will be greatly boosted as this process is reduced. For many, it's the concentration aspect that prevents them from succeeding with the willpower method, they could put up with the irritability and bad temper, but the failure to concentrate on something difficult once their crutch is removed ruins many.

Loss of concentration that users suffer when trying to escape isn't due to the absence of sex, let alone porn. You have mental blocks when you're addicted to something and when you have a mental block, what do you do? You fire up the browser, which doesn't cure the block so then what do you do? You do what you have to do, getting on with it just as non-users do.

When you're a user nothing is blamed on the cause, users never have erectile dysfunction, just occasional down time. The moment you stop using, everything that goes wrong is blamed on the reason you stopped. Now when you have a mental block, instead of just getting on with it, you begin to say \textit{"If only I could check my harem now, it would solve all my problems"}. You then begin to question your decision to quit and escape from the slavery.

If you believe that porn is a genuine aid to concentration, worrying about it will guarantee that you won't be able to concentrate. Doubting, not the physical withdrawal pangs creates the problem. Always remember, it's the user who suffers pangs, not non-users.

\section{Relaxation}

Most users think that porn helps to relax them, it doesn't. The frantic search to get the fix in those 'dark alleys of the internet' and the internal struggle straining at the leash to cross the red line doesn't sound like a very relaxing activity.

As night rolls in after a trip to a new place or a long day, we sit down to relax, relieving our hunger, thirst and are completely satisfied. The user isn't having another hunger to satisfy. Thinking of porn as the icing on the cake, but in actuality it's the 'little monster' that needs feeding. The truth is that the addict can never be completely relaxed and going through life it gets worse. Take one online comment from an ex-user: \\
  \textit{"I really believed that I had an evil demon in my make up, I now know that I had, however it wasn't some inherent flaw in my character but the little internet porn monster that was creating the problem. During those times I thought I had all the problems in the world, but when I look back on my life I wonder where all the great stress was. In everything else in my life I was in control, only thing controlling me was porn slavery. The sad thing is that even today I can't convince my children that it was the slavery that caused me to be so irritable."}

Every time I hear a porn addict trying to justify their addiction the message is, \textit{"Oh it helps me to relax."} On the internet I read about a single dad whose six year old son wanted to share his bed in the night after a scary movie, but the dad would refuse so that he could have his session and edge for hours.

Here's another smoking analogy, a couple of years ago adoption authorities threatened to prevent smokers from adopting children. A man rang up, irate. \textit{"You're completely wrong"}, he said \textit{"I can remember when I was a child, if I had a contentious matter to raise with my mother, I would wait until she lit a cigarette because she was more relaxed then."} Why couldn't the man talk to his mother when she wasn't smoking a cigarette?

Why are some users are so stressed when they're not getting their fix, even after real sex? One story online details a man working in the advertising field having 9's and 10's open for dates at any time, but losing interest in taking them out for dinner as internet porn was far easier, no restaurant spending or the possibility of a 'no' from his date at the end of an evening. Why be bothered when his little monster keeps him craving the low risk, high reward scheme at his fingertips upon reaching home?

Why are non-users completely relaxed then? Why are users not able to relax without a fix for a day or two? Read about the experience of a user taking the abstinence oath and quitting and you'll notice the struggle with temptations, clearly not relaxed at all when not allowed to have the 'only pleasure' they are 'entitled to enjoy'. They've forgotten what it's like to be completely relaxed. Porn can be likened to a fly being caught in a pitcher plant, to begin with the fly is eating the nectar but at some imperceptible stage the plant begins to eat the fly.

Isn't it time you climbed out of the plant?

\section{Social Night Sessions}

This is misinformation that seems to make sense, but doesn't. In order to control your appetite, will you eat at home before leaving to go to a restaurant or party? This is what you're doing with sessions before social nights, looking tired and not up to your best. The widespread adoption of pick-up techniques has introduced pressure to perform, pick-up and score. Attempting to drown your butterflies with porn and substances will only make the problem worse in the long run. Personally, I like a bit of anxiety to keep me focused and engaged and tiring yourself out mentally and physically with orgasm isn't going to help.

Social night porn is occasioned by two or more of our usual reasons for pleasure/prop seeking, social functions at their core are both stressful and relaxing. This might appear to be a contradiction but any form of socialisation can be stressful, even with friends, wanting to be yourself and completely relaxed. There's many occasions that have multiple factors present at any one time, take driving as an example since after all, your life is at stake. Stressful with concentration required for a sustained period of time, you need not be aware of these factors, your subconscious already receiving the message. By the same token being stuck in traffic jams or bored on a long highway drive, perhaps promising yourself a session upon reaching home.

Another good example is going on a first date, your mind throwing out questions about the person you're about to meet. Then meeting the person in the flesh, if enthusiasm starts to fade you'll start to feel too relaxed and feeling guilty for feeling this way. The tug of war has started, \textit{"I want sex or get me out of here ASAP"}, priming you for post date porn.

Even if the date went well and hours later you're back at their place, no matter which way it goes you won't be satisfied if your only goal is seeking orgasm. At other times you drive home alone, only thought being your online harem instead of congratulating yourself for your efforts. You can bet that someone in this position will have a session upon reaching home, it's often after nights like these when waking to feel uneasy emptiness, are the ones we'll miss the most when contemplating stopping porn. We think that life will never be quite as enjoyable again. In fact it's the same principle at work: the sessions simply provide relief from the withdrawal pangs, at some times having greater needs than others, greasing the water slide for the next cue.

Make this clear - It's not internet porn and harem dwellers that are special, it's the occasion. Once the need for porn is removed, such occasions will become more enjoyable and stressful situations less stressful.

\end{document}
