\documentclass[easypeasy.tex]{subfiles}
\chapter{Timing}
\begin{document}

Apart from the obvious point that it's doing you no good and that now is the right time to stop, timing is important. Society treats internet porn flippantly as a slightly distasteful habit that doesn't injure your health. This is untrue. It's  drug addition, a disease and destroyer of relationships in society. The worst thing that happens in most PMOer's lives is getting hooked on this awful addiction. If they stay hooked, horrendous things happen. Timing is therefore important to give yourself the right to a proper cure.

Firstly, identify the times or occasions when PMO appears to be important to you. If you're a businessperson and use it for the illusion of stress relief, pick a relatively slack period or a holiday. If you PMO mainly during boring or relaxing periods, doing the opposite. Regardless, take the attempt seriously and make it the most important thing in your life.

Look ahead for a period of three weeks and try to anticipate any event that might lead to failure. Occasions like a conference trip, partner out of town, ect, need not deter you, providing you anticipate them in advance and don't feel that you'll be deprived. Don't attempt to cut down in the meantime as this will only create the illusion that being denied is enjoyable. In fact, it helps to force yourself to watch and have as many PMO sessions as possible. While you are having the last session and your last time, be mindful of the disappointment due to satiation, unfulfilled expections, any bodily pain, withdrawal effects, peevishness and melancholy. Think of how marvellous it will be when you allow yourself to stop doing it.

{\huge WHATEVER YOU DO, DON'T FALL INTO THE TRAP OF JUST SAYING, "NOT NOW, LATER", AND PUTTING IT OUT OF YOUR MIND. WORK OUT YOUR TIMETABLE NOW AND LOOK FORWARD TO IT.}

Remember, you aren't giving anything up. On the contrary, you are about to receive marvellous positive gains.

For years, the medical profession has viewed porn as harmless without knowing the difference between the tame static porn of yesteryear and the latest VR streaming porn. The problem is that although every PMOer uses internet porn purely to relieve the dopamine craving, it's not the addiction to the chemical that hooks the PMOer, but self-brainwashing that results from the addiction. An intelligent person will fall for a confidence trick, but only a fool would continue falling for it when they realise it's a trick. Fortunately, most PMOers aren't fools; they only think they are. Each individual PMOer has their own private brainwashing. That's why there appears to be such a diverse range of PMOer types, only serving to compound the mysteries.

While the benefit of the original book was to quit smoking, which dealt with nicotine addiction (one of the quickest and addictive drugs known to man) and the personal logs collected from Reddit, NoFap and YBOP blogs and forums, I was agreeably surprised to realise that the philosophy propounded in the original book was still sound. The accumulated knowledge and challenge that Allen Carr and myself undertook is how to communicate that knowledge to each individual user. The fact I know every PMOer can not only find it easy to stop, but can actually enjoy the process is not only pointless but exceedingly frustrating unless I can make the PMOer realise it. Allen Carr in his original book explains his controversial advice:

  \textit{"Many people have said to me: 'You say, "Continue to smoke until you finish the book" This tends to make the PMOer take ages to read the book or just not finish it. Period. Therefore, you should change the instruction.' This sounds logical, but I know if the instruction were: 'Stop immediately', some smokers wouldn't even start reading the book. I had a smoker consult me in the early days. He said, 'I really resent having to seek your help, I know I'm strong-willed. In every other area of my life I'm in control. Why is it that all these other smokers by using their own willpower, yet I have to come to you?' He continued, 'I think I could it on my own, if I could smoke while I was doing it.'"}

This might sound like a contradiction but I know what the man meant. We think of stopping smoking as something that is very difficult to do. What do we need when we have something difficult to do? We need our little friend. So stopping smoking appears to be a double blow, not only do we have a difficult task we need to perform, which is hard enough, but the crutch we normally rely on such occasions is no longer available. But the real beauty of this method is that you don't need to give up while you go through the process of stopping. Getting rid of all your doubts and fears first so that when you finish the final cigarette you are already a non-PMOer and can enjoy being one.

So this hackbook on PMO will keep the same advice intact. No matter how much it's said that it will be easy, there will be a vast majority who will not be able to accept it due to their personal brainwashing on how difficult it is to quit.

The only question that has caused me to question the original advice seriously is this chapter on the matter of right timing. Above all, I advise that if your special occasions are stressful situations at the office then pick a holiday to make an attempt and vice versa. In fact, this isn't the easiest way, picking instead what you consider to be the most \textit{difficult} time instead. Whether that be stress, social, concentration or boredom, once you've proved that you can cope with and enjoy life in the worst possible situation, every other situation becomes easy. But if that was the definite instruction, would you even make the attempt?

Here's an analogy, my wife and I intend to swim together. We arrive at the pool at the same time, but rarely end up swimming together. The reason being that she immerses one toe and half an hour later is actually swimming. That's slow torture, I know in advance that at some stage, no matter how cold the water is, I'll have to brave it at some point. So I've learned to do it the easy way: Diving straight in. Now, assuming I was in a position to insist that if she didn't dive straight in, she wouldn't swim at all. Do you see the problem?

From feedback, I know that many PMOers have used the original advice given on timing to delay what they think will be the 'evil day'. My next thoughts were to use the technique used for the section on the advantages of PMOing, something along the lines of: \textit{"Timing is very important and in the next chapter you will be advised on the best time for you to make the attempt."} You turn the page and there's just a huge \textit{"NOW!"} That is, in fact, the best advice, but would you take it? This is the most subtle aspect of the porn trap. When we have genuine stress in our lives, it's not the time to stop yet, but when we have no stress, we have no desire to stop. Ask yourself the following questions:

\begin{enumerate}
  \item When you got onto porn for the first time, did you really decide then that you would continue to depend on it for the rest of your life without ever being able to stop? \textbf{Of course you didn't!}

  \item Are you going to continue the rest of your life without ever being unable to stop? \textbf{Of course you aren't!}
  \end{enumerate}

So when will you stop? Tomorrow? Next year? The year after? Isn't that what you've been asking yourself since you first realised you were hooked? Are you hoping that one morning you will wake up and just not want to PMO anymore? Stop kidding yourself, with any addiction you get progressively more hooked, not less. You think it will be easier tomorrow? Are you going to wait until you've actually started to feel that getting our of bed is harder than just masturbating? That would be a bit pointless.

The real trap is the belief that now isn't the right time - it will always be easier tomorrow. We believe that we live stressful lives, but in fact, we don't. We've taken most genuine stress out of our lives. When you leave home you don't live in fear of being attacked by wild animals, most don't wonder where our next meal will come from or if we'll have a roof over our heads tonight. Think of the life of a wild animal, never time a rabbit comes out of it's burrow, it's facing Vietnam for it's entire life. But the rabbit can handle it, it's got adrenaline and other hormones and so have we. The truth is, the most stressful periods for any creature are early childhood and adolescence. But three billion years of natural selection has equipped us to cope with stress, many people who have had hard childhoods grow up to lead normal lives.

It's a cliché to say, \textit{"If you haven't got your health, you've got nothing"} but it's absolutely true. When you feel physically and mentally strong you can enjoy the highs and handle the lows. We confuse responsibility with stress, responsibility only becoming stressful when we don't feel strong enough to handle it. What destroys most isn't the stresses, jobs or old age, but the lying crutches they turn to which are just illusions.

Look at it this way, you've already decided that you are not going to stay in the trap for the rest of your life. Therefore at some point whether you find it easy or difficult, you'll have to go through the process of getting free. PMOing isn't a habit or pleasure, it's drug addiction and a disease. We've already established that far from being easier to stop tomorrow, it'll get progressively harder. With a disease that's going to get progressively worse, the time to get rid of it is \textbf{now} - or as near as you can manage. Just think of how quickly each week of our lives comes and goes, that's all it takes. Think of how nice it will be to enjoy the rest of your life without ever-increasing black shadows hanging over you. If you follow all the instructions, you won't even have to wait three weeks or five days. You'll not only find it easy after closing down your browser: \textbf{You'll enjoy it!}
\end{document}
